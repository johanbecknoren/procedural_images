\documentclass[12pt]{article}
\usepackage{graphicx,url,placeins,algorithm,algorithmic,lipsum,amssymb,amsmath,subfigure,float}
\usepackage[hidelinks]{hyperref}
\usepackage[titletoc]{appendix}
\usepackage{pdfpages}
\usepackage{listings}
\usepackage{layout}
\usepackage[font=small,labelfont=bf]{caption}
\usepackage[utf8]{inputenc}
\graphicspath{{figures/}}
\setlength{\voffset}{0in}
\setlength{\headsep}{5pt}
\addtolength{\textheight}{2cm}
\setlength{\footskip}{5pt}
%
\begin{document}

\title{\vskip -6em Infinite Terrain in OpenGL \\ TNM084}   % type title between braces

\author{
 Johan Beck-Nor\'{e}n, johbe559@student.liu.se
 }
        \date{\today}    % type date between braces
        \maketitle
        
\begin{enumerate}
  \item Stefan Gustavssons implementation of 2D simplex noise in GLSL
  \item Statiskt vertex grid + statisk cam, camerapos som seed till noisefunc, displace i Y-led mha noise
  \item Ytnormaler för den displacade ytan beräknas mha finita differenser (partiella derivator) i vertex shadern.
  \item Implementerat i OpenGL och GLSL. GLEW och GLFW används för GL-context och fönsterhantering.
  \item Genom att hålla kameran statisk i förhållande till vertexgrid, men ändå göra translationsberäkningar på kameran,  så kan kamerapositionen skickas som uniform till VS och användas som seed till noise-funktionen. Eftersom noise-funtionen är deterministisk så kan vi förskjuta seeden med camerapos och på så sätt få illusionen att vi flyger över terrängen, när det i själva verket är vertices som endast förskjuts i höjdled.
  \item Färgsättningen av terrängen är simpel och skapas genom att tröska vertices i Y-led till ett visst värde i VS, alla normalerna beräknas dock som vanligt. I FS ges blå färg till de fragment under ett visst tröskelvärde, och grön färg annars.
  \item Simpel lighting model i FS med en ambient color och en directional ljuskälla
  \item Fractal Brownian Motion
  \item Nackdelar: - Det är dyrt att sampla noisefuntionen så ofta som jag gör, speciellt i och med de partiella derivatorna. Eftersom campos ändå skickas från CPUn hade det kanske varit mer performant att skapa en noise-textur på CPUn och ladda upp till GPUn istället.
  \item Försökte förgäves att implementera tessellation, men utan resultat. Kortet i min workstation stöder upp t.o.m OpenGL 4.3 men lyckades ej 
\end{enumerate}

\end{document}